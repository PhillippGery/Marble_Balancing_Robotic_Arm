\documentclass[journal]{IEEEtran}

\usepackage{cite}
\ifCLASSINFOpdf
\usepackage[pdftex]{graphicx}
\else
\usepackage[dvips]{graphicx}
\fi
\usepackage{amsmath}
\usepackage{array}
\hyphenation{op-tical net-works semi-conduc-tor}

\begin{document}
	
\title{Dynamic Spatial Stabilization and Trajectory Tracking of a Spherical Agent on a 6-DOF Robotic Platform}

	\author{Phillipp Gery (pgery), Kevin Biju Mathew (kb)%
		\thanks{College of Engineering, Purdue University, Indiana, USA.}}
	
	\markboth{Course Project Proposal: Robotics Control, February~2026}%
	{Gery \MakeLowercase{\textit{et al.}}: Autonomous Marble Balancing}
	
	\maketitle
	
	\begin{abstract}
		This project explores the integration of classical control theory and Reinforcement Learning (RL) to solve the problem of stabilizing a marble on a mobile platform. Using a UR7e 6-axis manipulator, we aim to maintain a marble's position at a target coordinate while the robot undergoes base-level or end-effector trajectories.
	\end{abstract}
	
	\begin{IEEEkeywords}
		ROS 2, Gazebo, LQR, Reinforcement Learning, UR7e, Trajectory Tracking.
	\end{IEEEkeywords}
	
	\section{Introduction}
	\IEEEPARstart{P}{recision} stabilization of unconstrained objects is a fundamental challenge in robotics. This project is different from traditional "Labyrinth" setups because it focuses on Active Surface Control. In this project, the robot must balance a marble on a board or follow a path on a board while also doing a second task, such as following a 3D path with its Endeffector.
	
	\section{System Platform and Simulator}
	For this project, we will utilize:
	\begin{itemize}
		\item \textbf{Robot Platform:} Universal Robots UR7e (6-DOF Manipulator).
		\item \textbf{Simulator:} Gazebo Classic 11. This physics-based simulator will provide ground-truth state data for the ball (position/velocity) and simulate the inertial dynamics of the UR7e and the board.
		\item \textbf{Software Stack:} ROS 2 Humble running on Ubuntu 22.04, with MATLAB for State feedback Controller design.
	\end{itemize}
	
	\section{Methodology and Learning Component}
	We propose a hybrid control strategy. An initial state-feedback controller (e.g., LQR) will be designed for baseline stabilization. 
	\par \textbf{Learning Component:} We will implement a Reinforcement Learning (RL) agent (using Proximal Policy Optimization or Twin Delayed DDPG) to refine the control policy. The agent will focus on learning the non-linear "residual" dynamics, such as rolling friction and surface-contact perturbations, which are difficult to model analytically.
	
	\section{Project Milestones}
	The following milestones outline the development trajectory:
	
	\subsection{Milestone 1: Planning and Control (10\%)}
	\textbf{Focus: Project Proposal + Control Component.}
	\begin{itemize}
		\item Derivation of the Nonlinear Model and the linearized State-Space model for the ball-and-plate system.
		\item \textbf{Trajectory Planning:} Development of a path planning module to generate, time-parameterized reference trajectories for the marble on the moving board.
		\item Development of the primary Marble Position Controller (LQR/State-Feedback).
		\item Implementation of the UR7e Inverse Kinematics (IK) solver in a ROS 2 node.
		\item Integration of the Robot, Board, and Marble URDF/Xacro into Gazebo with active joint controllers.
		\item Simulation tests with desired trajectories of the marble on the board.
	\end{itemize}
	
	\subsection{Milestone 2: Learning and Initial Results (10\%)}
	\textbf{Focus: Presentation + Learning Component.}
	\begin{itemize}
		\item Integration of a state observer (Kalman Filter) to handle Gazebo sensor noise and state estimation.
		\item Setup of the RL training environment (Gymnasium wrapper) for the Gazebo simulation.
		\item Initial training results showing the RL agent’s ability to reduce steady-state error compared to LQR alone.
		\item Presentation of preliminary results.
	\end{itemize}
	
	\subsection{Milestone 3: Full Project Report and Demo (15\%)}
	\textbf{Focus: Final Documentation + Full Demo.}
	\begin{itemize}
		\item Successful stabilization of the marble at target coordinates while the robot follows a 3D path.
		\item Comparative analysis between the baseline LQR and the Hybrid LQR+RL policy.
		\item Submission of a full IEEE-format technical report and final code repository.
		\item Demo videos showing the system resisting external manual perturbations in Gazebo.
	\end{itemize}
	

	
	
\end{document}